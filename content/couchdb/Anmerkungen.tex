Allgemeines:
Von Tobi:
- Wir sollten uns auf eine einheitliche Schreibweise von CouchDB einigen und die dann umsetzen (ich hab ein oder zwei Abweichungen gesehen)
- Wir sollten am Ende unsere Quellen überprüfen, weil wir evtl doppelte erstellt haben und dann auch doppelt in den Quellen stehen. 

Wegen der Struktur sollten wir uns einig werden, wäre es nicht sinnvoll unsere einleitende Sätze  die wir alle in unser Kapitel gemacht haben nach oben in die grße Introduction tun. Gleiche für Conclusion

Zu Leon's Texten:
Anmerkungen von Tobi:
- cURL utility -> Ich würde sagen, dass das die REST API ist, die mit cURL angesprochen kann (ganz am Anfang)
- Design Documents: Mir ist nicht so ganz klar, was das ist? Das Kapitel fängt damit an, was das enthält?
- Filters: Vielleicht bei den DVDs/CDs noch e.g. oder for example schreiben? Ich war einen Moment verwirrt
- Json: Ist der Autor Tim Juravich irgendjemand besonderes? Dann könnte man das vielleicht noch extra erwähnen?
- Vielleicht bei Json noch kurz einen Bezug zu CouchDB herstellen? Ich weiß zwar, wofür es gebraucht wird, aber irgendwie habe ich das Gefühl, dass ein paar wenige Sätze da noch hilfreich wären? Sonst "hängt" das so am Ende des Dokuments

meine Anmerkungen sind direkt als Kommentar drin :)

Zu Tobi's Texten:
Anmerkung von Ines:
bei Kapitel 1.3.2: Schreibt man StatusCodes nicht auseinander -> Status Codes
Zu Ines Texten:
Anmerkungen von Tobi:
- Creating server administrator, ziemlich am anfang: "This situation is a very big security grab" -> Meinst du gap? Ich kenne grab nicht und der Übersetzer gibt nichts wirklich gutes aus?
- Ich würde vielleicht überlegen noch eine kurze Zusammenfassung zu schreiben, auf mich wirkte das so, dass man viele Sicherheitseinstellungen bei der Einrichtung einer CouchDB konfigurieren muss? Das wäre halt als Zusammenfassung vielleicht ganz nett?




