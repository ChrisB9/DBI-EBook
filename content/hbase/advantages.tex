%!TEX root = ../dokumentation.tex

\subsection{Advantages and Disadvantages}
When evaluating the advantages and disadvantages of HBase it can be revealing to have a closer look at its features. According to \cite{hbase.achari.2015} HBase's characteristics can be named with the following key words: \textbf{sparse}, \textbf{distributed}, \textbf{multidimensional}, \textbf{map} and \textbf{sorted}. 

The word sparse seems to be a quite negative description for a database, at least in the german translation (dt.: dürftig). It sounds like HBase does not provide all of the important feautures a database should have. But actually the word sparse does not refer to the database itself but to the data stored within it. As described earlier the table schema can be absolutely flexible. The columns don't need to be the same in every row, two rows don't even need to be similar. For that reason there is no need to fill missing values with NULL to keep the schema consistent like it would be nessecary in relational databases. This saves storage space. Additionaly this structure has the advantage that columns can be added or removed at every moment of the database's lifetime. \cite{hbase.achari.2015}

Furthermore HBase is distributed to several nodes which can be either physically or virtually implemented. When using a physical implementation, the data is stored on several independent (physical) machines, \cite{hbase.wilson.2008} which are called \textit{Master Server} and \textit{Region Server}. Usually a cluster insists of several Region Servers which store the data \cite{hbase.vohra.2016} and at least one (up to nine) Master Servers which control the Region Servers within the cluster. \cite{hbase.apache.foundation.2017} This is HBase's auto failover which makes it reliable and protects against dataloss if one node in the cluster is failing \cite{hbase.wilson.2008} - on the condition that there has to be more than one Master Server, otherwise there’s the risk of a single point of failure if the Master Server fails. \cite{hbase.shriparv.2014}
Auto-sharding explains a bit more in detail how the data is stored on a Region Server. A region is the "basic unit of scalability and load balancing in HBase". And as already explained Chapter \ref{lblHBaseTechnologies} they simply contain contiguous ranges of rows. When these rows get to big for one Region Server they are split automatically by the system. This is called auto-sharding and permits the dristribution of data to different servers (Region Servers). This allows fast recovery after a server failed, as well as load fine-grained balancing because the regions can be moved between the Region Servers the way it is most suitable \cite{hbase.george.2011}

The terms multidimensional, map and sorted can be explained best in context. Map simply means the structure of key value pairs. Every value saved in the database has one specific key which consists of row key and timestamp among others. And these key value pairs are sorted alphabetically by key. This is important when designing the row key because at this point the administrator can ensure that relating values will be stored together within one region (e.g. two row keys of relating data: dhbw.2017.tinf14c and dhbw.2017.tinf14a). \cite{hbase.wilson.2008}
The conception of the key permits different versions for a value. Due to the timestamp whithin the key it is possible to overwrite an existing value without deleting it. Instead only the latest value is output on a table scan while the other versions are still available (see page \pageref{lblHBaseExample} Chapter \ref{lblHBaseExample}:  Update value). This is called multidimensional and gives the advantage of reproducing updates. \cite{hbase.achari.2015}

Additionally HBase has the advantage of the Apache Hadoop ecosystem with a lot of features especially for Big Data use cases, such as the MapReduce procedure for analytics. \cite{hbase.shriparv.2014}
The MapReduce procedure was just mentioned as an advantage but it also reveals a disadvantage. HBase does not handle JOINs like it is common in relational databases. Therefore the MapReduce layer provides a remedy. But as well it involves much more effort. A comprehensive knowledge about the Hadoop ecosystem would be necessary to use it effectively. And this fact actually applies to HBase in general. As it is not thought of being used as a standalone configuration the complexity of the Hadoop ecosystem is always there.
 As already mentioned when talking about sorted keys, it might also be a disadvantage that a row can only be indexed by its key whereas relational databases permit more options or using keys. \cite{hbase.shriparv.2014}

The HBase book \cite{hbase.apache.foundation.2017} gives advices when to use or when not to use the database. They definitively point out that it is not suitable for every problem. It is most important to have a huge amount of data. Huge amount in this context means hundreds of millions of rows, \textit{just} a few thousand or even million rows would not be enough to really get advantages from HBase. In this case it is suggested to stay with relational databases. This is for the reason that the cluster would not be made advantage of.
Then they stress that HBase is absolutely different from relational databases and most of the common features there are not available in HBase, such as data types, secondary indexes, transactions and advanced query languages. Therefore it is not possible to just migrate a relational database to HBase it is rather a complete redesign.
And finally they mention it is important to have enough hardware (minimum about five nodes) which is at least needed for the Hadoop filesystem. Indeed HBase can run standalone on a computer but this case should just be used for development cases. \cite{hbase.apache.foundation.2017}

Based on the investigation made for this chapter it seems like HBase provides more advantages than disadvantages. This should not be taken as a full fact but rather than a clue. When taking a decision for a NoSQL database it is advisable to compare these directly. This chapter mostly names advantages and disadvantages compared to relational databases. Therefore it might rather be helping to decide wheter a relational database or HBase are more suitable for the use case.


 