\section{Limits}

RethinkDB has some limitations. 
Some are hard limitations and some are soft limitations.
\cite{RethinkLimits}

Shards are hard limited to 64.
the number of databases are not limited by anything than diskspace and RAM.
Likewise the number of tables.
The recommended limit for a single Document is 16MB in size.
This should be respected for preformance reasons.
Arrays in a RethinkDB Server have a limited length.
This length can be configured per query and is by default 100.000 elements.
Primary keys are limited to 127 characters.
At secondary keys there is no hard limit.
But if a secondary key exceeds its limit, RetinkDB will use linear search for the following characters.

A JSON queries are hard limited as well.
They can only be 64 MB in size or smaller.

The ordering in RethinkDB is byte-wise.
The functions orderBy and between uses the byte representaiton of the carater.
This has the result that RethinkDB dose not normalize identical characters with multiple codepoints.
For example the character "é" has the UTF-8 representation \code{\\u0065\\u0301} and \code{\\u00e9}, the will be grouped separately.

The usage of the cli option \code{--direct-io} is limited to file systems that support it.
Typically encrypted and compressed file systems will not support this option.

Numbers are double precision IEEE 754 floating point.
Integers are stored precisely from -2^53 to 2^53, outside that range they will be rounded.

\section{Data Types}

RetinkDB has all the basic Data types. 
Number, string, boolean, object, array and the \code{null} value. 
Additionaly RethinkDB stores specific data types like tables, streams, selections, binary objects, time objects, geometry data types, and grouped data.
\cite{RethinkDataTypes}

\textbf{Numbers} can bee Integer or Floatingpoint values. 
They will be stored with 64-bit percision.
\code{NaN} or Infinit can not be stored.

\textbf{String} are stored with UTF-8 encoding.

\textbf{Booleans} can be \code{true} or \code{false}

\textbf{Null} is a special value.
It is not the number zero or an empty string.
It symbolizes the absence of any other value.

\textbf{Objects} are key-value pairs.
Any valid JSON object is a valid RethinkDB object.
\begin{lstlisting}[frame=single, caption=example Object, label=refdoc]
{
	"key"  : "valueString",
	"key2" : false,
}
\end{lstlisting}
The keys can only be strings, but the values can be any data type.
A RethinkDB Document is a object.

\textbf{Arrays} are lists of values.
Arrays can be empty.
It is not enforced to use only on type of values in an Array.
But it is highly recommended.
\begin{lstlisting}[frame=single, caption=example Arrays, label=refdoc]
[
	"valueString",
	false,
]
[]
\end{lstlisting}
The values of an Array can be any data type.
If you use arrays for many values, more than 100.000 you should notice that arrays are fully loaded into Server RAM before send to the client.

\\
Specific Data types

\textbf{Databases} are returned after a \code{db} function call.

\textbf{Tables} are returned after a \code{getAll} function call.
Their behavior is similar to selections.

\textbf{Selections} are the result of \code{filter} ot \code{get} function calls.
Selections comes in three variants \code{Selection<Object>}, \code{Selection<Array>} and \code{Selection<Stream>}.
Selctions are wirteable.

\textbf{Streams} are like arrays, lists.
Streams are not writable.
Streams are lazy, they give you the next dataset if the current has been processed.
With Streams you can read tremendous long lists of Data, without holding everything in Memory.


\section{Changefeeds}

Changefeeds are the main part of RethinkDB's real-time functionality.
Almost all queries can be used as a changefeed.
With a Changefeed you recive any changes on your query.

\begin{lstlisting}[frame=single, caption=example usage, label=refdoc]
r.table('project').changes().run(conn, (err, cursor) => {
	cursor.each((err, row) => {
		if(err) throw err;
		processRow(row);
	})
});
\end{lstlisting}

The \code{changes()} method returns a cursor.
The \code{each()} method of that cursor object iterates over each Change.
If the underlaying data of the query changes the callback of each will be called.
This iteration will be infinit.
For example a new project will be added \code{{id:3, name: "awesome Project", priority:1}} into the project table.
The \code{each()} callback will be called with the changes of the table.

\begin{lstlisting}[frame=single, caption=example usage, label=refdoc]
{
	old_val: null,
	new_val: {id:3, name: 'awesome Project', priority:1}
}
\end{lstlisting}

If the callback for \code{each} returns false, the cursor will stop iterating.

\textbf{single document changefeeds}

A single docuemtn changefeed returns only changes to the document.

\begin{lstlisting}[frame=single, caption=example usage, label=refdoc]
r.table('project').get(3).changes().run(conn, callback);
\end{lstlisting}

This would listen for changes of the project with id 3.

\textbf{Filtering Changefeeds}

The method \code{changes()} integrates with the ReQL language.
\code{changes()} can be used with mostly any outher command.

The chaning of ReQL methods can be as complex as you wish.

\begin{lstlisting}[frame=single, caption=example usage, label=refdoc]
r.table('project').changes().filter(
    r.row('new_val')('priority').lt(r.row('old_val')('priority'))
)('new_val').run(conn, callback)
\end{lstlisting}

This will be notified on every change, there the priority is lower as the previous.

Chaning with changefeeds has some limitations.
Please read the current documentation for these limitations.

\textbf{handling latency}

With some latency on the client side, it is possible that there will be mutliple writes into a table, before the clinet gets all changefeed events.
By defualt the subscriber of the changefeed will only get one change object.
If this is not the desired procedure the option \code{squash: false} should be used on the \code{changes()} method.
With this option the change object will be send for every change.

Changefeeds do not have a delivery guarantee.

